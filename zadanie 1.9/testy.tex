\documentclass[12pt,a4paper]{article}

% ustawienia marginesu
\usepackage[left=1.6in,right=.8in,top=1.5in,bottom=1.5in]{geometry}

% polskie reguły dzielenia wyrazów itd
\usepackage{polski}

% polskie znaki zakodowane w UTF8
\usepackage[utf8]{inputenc}

% automatyczne generowanie odnośników w plikach PDF
\usepackage[pdftex,linkbordercolor={0 0.9 1}]{hyperref}

% pakiety matematyczne
\usepackage{amsthm,amsmath,amsfonts,amssymb,mathrsfs,amsmath}

% ładne składanie odnośników do stron www
\usepackage{url}

% rozbudowane możliwości wypunktowań
\usepackage{enumerate}

% możliwość dodawania plików graficznych
\usepackage{graphicx} 

%%% definicje twierdzeń itd :)
\newtheorem{tw}{Twierdzenie}[section]
\newtheorem{stw}[tw]{Stwierdzenie}
\newtheorem{fakt}[tw]{Fakt}
\newtheorem{lemat}[tw]{Lemat}

\theoremstyle{definition}
\newtheorem{df}[tw]{Definicja}
\newtheorem{ex}[tw]{Przykład}
\newtheorem{uw}[tw]{Uwaga}
\newtheorem{wn}[tw]{Wniosek}
\newtheorem{zad}{Zadanie}

% oznaczenia zbiorów liczbowych
\DeclareMathOperator{\R}{\mathbb{R}}
\DeclareMathOperator{\Z}{\mathbb{Z}}
\DeclareMathOperator{\N}{\mathbb{N}}
\DeclareMathOperator{\Q}{\mathbb{Q}}


% wartość bezwzględna, norma, iloczyn skalarny, nośnik, rozpięcie przestrzeni...
\providecommand{\abs}[1]{\left\lvert#1\right\rvert}
\providecommand{\var}[1]{\operatorname{var}(#1)}

% fajne nagłówki i stopki na stronie
\usepackage{fancyhdr}
\pagestyle{fancy}
\fancyhf{}
\fancyfoot[R]{\textbf{\thepage}}
\fancyhead[L]{\small\sffamily \nouppercase{\leftmark}}
\renewcommand{\headrulewidth}{0.4pt} 
\renewcommand{\footrulewidth}{0.4pt}




\title{Testy}
%\author{Maciej Stankiewicz}

%\date{30 listopada 2010}

\begin{document}
%\maketitle
%\tableofcontents


\section{Test 1, wieczorem dopisze reszte tylko fajnie jakby ktoś to potem sformatował}

	\begin{displaymath} 
\begin{split}
		n=2; a=1; b=3  \\
		x_0=1; x_1=2; x_2=3\\
l_0=\prod\limits_{k=0;k\neq0}^2\frac{x-x_k}{x_0-x_k}=\frac{x-x_1}{x_0-x_1}*\frac{x-x_2}{x_0-x_2}=\frac{x-2}{-1}*\frac{x-3}{-2}=\frac{1}{2}x^2-\frac{5}{2}x+3\\
l_0(x_0)=l_0(1)=1; l_0(x_1)=l_0(2)=0; l_0(x_2)=l_0(2)=0;\\
l_1=\prod\limits_{k=0;k\neq1}^2\frac{x-x_k}{x_1-x_k}=\frac{x-x_0}{x_1-x_0}*\frac{x-x_1}{x_0-x_2}=\frac{x-1}{1}*\frac{x-3}{-1}=-x^2+4x-3\\
l_1(x_0)=l_1(1)=0; l_1(x_1)=l_1(2)=1; l_1(x_2)=l_1(2)=0;\\
l_2=\prod\limits_{k=0;k\neq2}^2\frac{x-x_k}{x_2-x_k}=\frac{x-x_0}{x_2-x_0}*\frac{x-x_2}{x_0-x_1}=\frac{x-1}{2}*\frac{x-2}{1}=\frac{1}{2}x^2-\frac{3}{2}x+1\\
l_2(x_0)=l_2(1)=0; l_2(x_1)=l_2(2)=0; l_2(x_2)=l_2(2)=1;\\
1*\\
A_0=1; A_1=4; A_2=9\\
L(x)=\sum\limits_{i=0}^2A_i*l_i(x)=1(\frac{1}{2}x^2-\frac{5}{2}x+3)+4(x^2+4x-3)+9(\frac{1}{2}x^2-\frac{3}{2}x+1)=x^2\\
2*\\
A_0=2; A_1=1; A_2=3\\
L(x)=\sum\limits_{i=0}^2A_i*l_i(x)=2(\frac{1}{2}x^2-\frac{5}{2}x+3)+1(x^2+4x-3)+3(\frac{1}{2}x^2-\frac{3}{2}x+1)=\frac{3}{2}x^2-\frac{11}{2}x+6\\
\end{split}
\end{displaymath}

\section{Test 2}
\begin{displaymath} 
\begin{split}
n=1; a=1; b=2\\
\end{split}
\end{displaymath}
\section{Test 3}
\begin{displaymath} 
\begin{split}
n=3; a=0; b=6
\end{split}
\end{displaymath}
\end{document}
